\documentclass{article}
\usepackage[utf8]{inputenc}
\usepackage[margin=1in]{geometry}

\usepackage{enumerate}
\usepackage{amsmath, amssymb, amsfonts}
\usepackage{physics}
\usepackage{mathtools} % align matrix elements

% mathematical functions that should appear as non-italic in math environments
\newcommand{\nullspace}{\mathrm{null}}
\newcommand{\rref}{\mathrm{rref}}
\newcommand{\col}{\mathrm{col}}
\newcommand{\proj}{\mathrm{proj}}
\newcommand{\Span}{\mathrm{span}}
\renewcommand{\rank}{\mathrm{rank}}
\newcommand{\Cov}{\mathrm{Cov}}
\newcommand{\Var}{\mathrm{Var}}
\renewcommand{\Trace}{\mathrm{Trace}}

% useful math functions
\newcommand{\E}{\mathbb{E}}
\renewcommand{\P}{\mathbb{P}}
\newcommand{\Normal}{\mathcal{N}}
\newcommand{\1}[1]{\mathbf{1}\left\{#1\right\}}
\newcommand{\sumn}[1]{\sum_{#1=1}^n}

% image setup
\usepackage{graphicx}
% replace with whatever directory you use for images
\graphicspath{{./images/}}
\usepackage{float}
\usepackage{adjustbox}

% some really basic syntax highlighting
\usepackage{listings}
\usepackage{xcolor}
\definecolor{grape}{rgb}{0.35, 0.29, 0.49}
\lstset{backgroundcolor=\color{white},
        basicstyle=\footnotesize\ttfamily,
        showstringspaces=false,
        commentstyle=\color{gray},
        keywordstyle=\color{teal},
        stringstyle=\color{grape}}


\begin{document}
\textbf{Collaborators:}
\section*{Conceptual Questions}
\begin{enumerate}[1.]
% 0.
\item \begin{enumerate}[a.]
      % 0.a
      \item
      % 0.b
      \item
      % 0.c
      \item
      % 0.d
      \item
      % 0.e
      \item
      % 0.f
      \item
      % 0.g
      \item
      \end{enumerate}
\end{enumerate}

\section*{Convexity and Norms}
\begin{enumerate}[1.]
\setcounter{enumi}{1}
% 1.
\item \begin{enumerate}[a.]
      % 1.a
      \item
      % 1.b
      \item
      \end{enumerate}
% 2.
\item
% 3.
\item \begin{enumerate}[a.]
      % 3.a
      \item
      % 3.b
      \item
      % 3.c
      \item
      % 3.c
      \item
      \end{enumerate}
% 4.
\item
\end{enumerate}

\section*{Lasso on a real dataset}
\begin{enumerate}[1.]
\setcounter{enumi}{4}
% 5.
\item \begin{enumerate}[a.]
      % 5.a
      \item
      % 5.b
      \item
      % 5.c
      \item
      \end{enumerate}
% 6.
\item \begin{enumerate}[a.]
      % 6.a
      \item
      % 6.b
      \item
      \setcounter{enumii}{0}
      % 6.a (again?)
      \item
      % 6.b
      \item
      % 6.c
      \item
      % 6.d
      \item
      % 6.e
      \item
      \end{enumerate}
% 7.
\item \begin{enumerate}[a.]
      % 7.a
      \item
      % 7.b
      \item \begin{enumerate}[i.]
            % 7.b.i
            \item
            % 7.b.ii
            \item
            \end{enumerate}
      % 7.c
      \item
      % 7.d
      \item
      \end{enumerate}
\end{enumerate}
\end{document}
